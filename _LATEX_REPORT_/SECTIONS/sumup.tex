\documentclass[../main.tex]{subfiles}
\LoadClass[a4paper,12pt]{article}
\documentclass{article}
%%%%%%%%%%%%%%%%%%%%%%%%%%%%%%%%%%%%%%%%%%%%%%%%%%%%%%%%%%%%%%%
%                        Packages import                      %
%%%%%%%%%%%%%%%%%%%%%%%%%%%%%%%%%%%%%%%%%%%%%%%%%%%%%%%%%%%%%%%
%%%%%%%%%%%%%%%%%%%%%%%%%%%%%%%
%     import des packages     %
%%%%%%%%%%%%%%%%%%%%%%%%%%%%%%%
\usepackage{algorithm}
\usepackage{algpseudocode}
\usepackage{amsmath,amsfonts,amssymb}
\usepackage{anyfontsize}
\usepackage{array}
\usepackage[english]{babel}
\usepackage{colortbl}
\usepackage{comment}
\usepackage{enumitem}
\usepackage{eqnarray}
\usepackage{eso-pic}
\usepackage{fancybox}
\usepackage{fancyhdr}
\usepackage{float}
\usepackage[T1]{fontenc} 
\usepackage{fourier-orns}
\usepackage{gensymb}
\usepackage{geometry}
\usepackage{graphicx}
\usepackage{hyperref}
\usepackage[xindy,toc,nonumberlist,acronym]{glossaries}
\usepackage{ifnextok}
\usepackage{ifthen}
\usepackage{import}
\usepackage{indentfirst}
\usepackage[utf8]{inputenc}
\usepackage{lastpage}
\usepackage{libertine}
\usepackage{lipsum}
\usepackage{listings}
\usepackage{mathtools}
\usepackage{mdframed}
\usepackage{multicol}
\usepackage{multirow}
\usepackage{pdfpages}
\usepackage{pifont}
\usepackage{siunitx}
\usepackage{stmaryrd}
\usepackage{subcaption}
\usepackage{subfiles}
\usepackage{tabularx}
\usepackage{tcolorbox}
\usepackage{textcomp}
\usepackage{ulem}
\usepackage{wrapfig}


%%%%%%%%%%%%%%%%%%%%%%%%%%%%%%%%%%%%%%%%%%%%%%%%%%%%%%%%%%%%%%%
%                                                             %
%                       Codes & sections                      %
%                                                             %
%%%%%%%%%%%%%%%%%%%%%%%%%%%%%%%%%%%%%%%%%%%%%%%%%%%%%%%%%%%%%%%

% Création des boolean
\newboolean{intro}
\newboolean{sumup}
\newboolean{annexe}
\newboolean{biblio}
\newboolean{abst}
\newboolean{thx}
\newboolean{gloss}
\newboolean{fortran}
\newboolean{cpp}
\newboolean{python}
\newboolean{ffpp}

%% Sections %%
% Renseigner si le Rapport contient une introdcution 
\setboolean{intro}{true}
% Renseigner si le Rapport contient une conclusion 
\setboolean{sumup}{true}
% Renseigner si le Rapport contient des annexes 
\setboolean{annexe}{false}
% Renseigner si le Rapport contient un abstract
\setboolean{abst}{false}
% Renseigner si le Rapport contient des remerciements
\setboolean{thx}{false}
% Renseigner si le document contient une bibliographie
\setboolean{biblio}{false} 
% Renseigner si le document contient un glossaire
\setboolean{gloss}{false}

%% Codes %%
% Renseigner si le Rapport contient des codes fortran
\setboolean{fortran}{false}
% Renseigner si le Rapport contient des codes C++
\setboolean{cpp}{false}
% Renseigner si le Rapport contient des codes Python
\setboolean{python}{false}
% Renseigner si le Rapport contient des codes FreeFEM++
\setboolean{ffpp}{false}

% =============================================================
\ifthenelse{\boolean{fortran}}{ % Mise en page des codes F90
%%%%%%%%%%%%%%%%%%%%%%%%%%%%%%%%%%%%%%%%%%%%
% Mise en page des lignes de codes Fortran %
%%%%%%%%%%%%%%%%%%%%%%%%%%%%%%%%%%%%%%%%%%%%
\definecolor{codebackground}{rgb}{0.1,0.1,0.1}  % Couleur de fond grise sombre
\definecolor{codecomment}{rgb}{0.7,0.7,0.7}    % Couleur grise très claire pour les commentaires

\lstdefinelanguage{myF90}
{
  language=Fortran,
  morekeywords={allocate,case,data,do,else,elseif,end,endif,for,function,if,module,program,return,select,subroutine,then,where},
  morecomment=[l]!--,
  morecomment=[l]!>,
  morecomment=[s]{/*}{*/},
  morestring=[b]',
}

\lstdefinestyle{f90}{
  language=Fortran90,
  basicstyle=\ttfamily,
  keywordstyle=\bfseries\color{blue},
  commentstyle=\itshape\color{green!60!black},
  stringstyle=\color{orange},
  numbers=left,
  numberstyle=\tiny\color{gray},
  stepnumber=1,
  numbersep=5pt,
  frame=single,
  showspaces=false,
  showstringspaces=false,
  breaklines=true, % Permet la césure des lignes
  breakatwhitespace=false, % Permet les sauts de page uniquement aux espaces
  tabsize=4,
  captionpos=b,
}}


% =============================================================
\ifthenelse{\boolean{cpp}}{ % Mise en page des codes C++
%%%%%%%%%%%%%%%%%%%%%%%%%%%%%%%%%%%%%%%%
% Mise en page des lignes de codes C++ %
%%%%%%%%%%%%%%%%%%%%%%%%%%%%%%%%%%%%%%%%
\lstdefinelanguage{myCpp}{
  language=C++,
  morekeywords={class, public, private, protected, virtual, const, override, template, typename, constexpr},
  morecomment=[l]{//},
  morecomment=[s]{/*}{*/},
  morestring=[b]",
  keywordstyle=\bf\color{blue!80!white},
  stringstyle=\it\color{orange!80!red},
  commentstyle=\color{Green},
}

\lstdefinestyle{cpp}{
  language=myCpp
  basicstyle=\fontsize{10}{11}\ttfamily,
  numbers=left,
  numberstyle=\tiny\color{gray},
  stepnumber=1,
  numbersep=5pt,
  frame=single,
  showspaces=false,
  showstringspaces=false,
  breaklines=true,
  breakatwhitespace=true,
  tabsize=4,
  captionpos=b,
}}


% =============================================================
\ifthenelse{\boolean{python}}{  % Mise en page des codes Python
%%%%%%%%%%%%%%%%%%%%%%%%%%%%%%%%%%%%%%%%%%%
% Mise en page des lignes de codes Python %
%%%%%%%%%%%%%%%%%%%%%%%%%%%%%%%%%%%%%%%%%%%
\lstdefinelanguage{myPython}
{
  language=Python,
  morekeywords={if, else, while, for, def, import, class, and, or, not, True, False},
  morecomment=[l]{"},
  morecomment=[l]{\#},
  morecomment=[s]{'''}{'''},
  morestring=[b]',
  morestring=[b]""",
}

\lstdefinestyle{python}{
  language=myPython,
  basicstyle=\fontsize{10}{11}\sffamily,
  keywordstyle=\bf\color{blue!80!white},
  stringstyle=\it\color{orange!80!red},
  commentstyle=\color{Green},
  numbers=left,
  numberstyle=\tiny\color{gray},
  stepnumber=1,
  numbersep=5pt,
  frame=single,
  showspaces=false,
  showstringspaces=false,
  breaklines=true,
  breakatwhitespace=false,
  tabsize=4,
  captionpos=b,
}
}


% =============================================================
\ifthenelse{\boolean{ffpp}}{  % Mise en page des codes FreeFEM++
%%%%%%%%%%%%%%%%%%%%%%%%%%%%%%%%%%%%%%%%%%%%%%
% Mise en page des lignes de codes FreeFEM++ %
%%%%%%%%%%%%%%%%%%%%%%%%%%%%%%%%%%%%%%%%%%%%%%
\lstdefinelanguage{myFFpp}{
  language=C++,
  morekeywords={mesh, solve, varf, fespace, macro, real, border, int2d, int1d},
  morecomment=[l]{//},
  morecomment=[s]{/*}{*/},
  morestring=[b]",
  keywordstyle=\bf\color{red!70!black},
  stringstyle=\it\color{purple!80!blue},
  commentstyle=\color{Green},
}

\lstdefinestyle{ffpp}{
  language=myFFpp,
  basicstyle=\fontsize{10}{11}\ttfamily,
  numbers=left,
  numberstyle=\tiny\color{gray},
  stepnumber=1,
  numbersep=5pt,
  frame=single,
  showspaces=false,
  showstringspaces=false,
  breaklines=true,
  breakatwhitespace=true,
  tabsize=4,
  captionpos=b,
}}


%%%%%%%%%%%%%%%%%%%%%%%%%%%%%%%%%%%%%%%%%%%%%%%%%%%%%%%%%%%%%%%
%                      Environments setup                     %
%%%%%%%%%%%%%%%%%%%%%%%%%%%%%%%%%%%%%%%%%%%%%%%%%%%%%%%%%%%%%%%
%%%%%%%%%%%%%%%%%%%%%%%%%%%%%%%%%%%%%%%%%%%%%%%%%%%%%%%%%%%%%%%
%                                                             %
%                      Environments setup                     %
%                                                             %
%%%%%%%%%%%%%%%%%%%%%%%%%%%%%%%%%%%%%%%%%%%%%%%%%%%%%%%%%%%%%%%
%%%%%%%%%%%%%%%%%%%%%%%%%%
% Mise en page des notes %
%%%%%%%%%%%%%%%%%%%%%%%%%%
\newcounter{notecounter}
\newenvironment{note}{%
  \refstepcounter{notecounter}
  \par\addvspace{\baselineskip}%
  \begin{tcolorbox}[
    colback=green!20,
    colframe=green!50!black,
    left=0.5em,
    right=0.5em,
    title=NOTE .\thenotecounter,
    fonttitle=\bfseries,
  ]%
    \itshape\fontsize{11pt}{12pt}\selectfont
}{%
  \end{tcolorbox}%
  \par\addvspace{\baselineskip}%
}

%%%%%%%%%%%%%%%%%%%%%%%%%%%%
% Mise en page des summary %
%%%%%%%%%%%%%%%%%%%%%%%%%%%%
\newenvironment{summary}{%
  \renewcommand{\thesubsection}{~\Roman{section}}
  \par\addvspace{\baselineskip}%
  \addcontentsline{toc}{subsection}{\protect\numberline{}SUMMARY}
  \begin{tcolorbox}[
    colback=red!20,
    colframe=red!80!black,
    left=0.5em,
    right=0.5em,
    title=SUMMARY \thesubsection,
    fonttitle=\bfseries,
  ]%
    \itshape\fontsize{11pt}{12pt}\selectfont
}{%
  \end{tcolorbox}%
  \par\addvspace{\baselineskip}%
}

%%%%%%%%%%%%%%%%%%%%%%%%
% Appendix environment %
%%%%%%%%%%%%%%%%%%%%%%%%
\newcounter{annexecounter}
\newenvironment{annexe}[1]{
  \refstepcounter{annexecounter}
  \noindent \large \bfseries A\theannexecounter~- #1%
  \addcontentsline{toc}{subsection}{\protect\numberline{}A\theannexecounter~- #1}
  \vspace{0.3cm}
}{%
    \par\addvspace{\baselineskip}%
}

%%%%%%%%%%%%%%%%%%%%%%%%%%%%%%%%%%%%%%%%%%%%%%%%%%%%%%%%%%%%%%%
%                       Custom commands                       %
%%%%%%%%%%%%%%%%%%%%%%%%%%%%%%%%%%%%%%%%%%%%%%%%%%%%%%%%%%%%%%%
%%%%%%%%%%%%%%%%%%%%%%%%%%%%%%%%%%%%%%%%%%%%%%%%%%%%%%%%%%%%%%%
%                                                             %
%                       Custom commands                       %
%                                                             %
%%%%%%%%%%%%%%%%%%%%%%%%%%%%%%%%%%%%%%%%%%%%%%%%%%%%%%%%%%%%%%%
%%%%%%%%%%%%%%%%%%%%%%%%%%%
% Declare custom commands %
%%%%%%%%%%%%%%%%%%%%%%%%%%%
\newcommand{\Romannumeral}[1]{\MakeUppercase{\romannumeral #1}} % make uppercased roman style number #1
\renewcommand{\div}{\operatorname{div}} % alphabetic divergent operator in math mode
\newcommand{\grad}{\operatorname{grad}} % alphabetic gradient operator in math mode
\renewcommand{\restriction}[2]{\left.#1\right|_{#2}}  % mathematical function #1 restriction to #2
\newcommand{\explain}[1]{\textit{-- #1 --}} % Command \explain for explanation, remplacing ()

%%%%%%%%%%%%%%%%%%%%%%%%%%
% SI units customisation %
%%%%%%%%%%%%%%%%%%%%%%%%%%
% \sisetup{output-exponent-marker=\ensuremath{\mathrm{e}}}

\DeclareSIUnit{\degc}{\degree C}
\DeclareSIUnit{\bara}{barA}
\DeclareSIUnit{\rpm}{rpm}
\DeclareSIUnit{\Wh}{Wh}
\DeclareSIUnit{\it}{\textit{it}}

%%%%%%%%%%%%%%%%%%%%%%%%%%%%%%%%%%%%%%%%%%%%%%%%%%%%%%%%%%%%%%%
%                                                             %
%                      Document settings                      %
%                                                             %
%%%%%%%%%%%%%%%%%%%%%%%%%%%%%%%%%%%%%%%%%%%%%%%%%%%%%%%%%%%%%%%
\numberwithin{figure}{section}
\numberwithin{table}{section}
%%%%%%%%%%%%%%%%%%%%%%%
% Titre, auteurs, ... %
%%%%%%%%%%%%%%%%%%%%%%%
\newcommand{\titre}{Modélisation du trimaran Zephyr\\-\\\vspace{0.5cm} Code de calcul CERF}
\newcommand{\sujet}{Interactions Fluide-Structure}
\newcommand{\sujets}{Seatech 3A - MOCA}
\newcommand{\auteur}{Fabien MATHE}
\newcommand{\auteurs}{Yohan STABLO}
\newcommand{\referent}{M. Frédéric GOLAY}

%%%%%%%%%%%%%%%%%%%%%%%%%%%%%%
% En-têtes en pieds de pages %
%%%%%%%%%%%%%%%%%%%%%%%%%%%%%%
\geometry{hmargin=2cm,vmargin=2.3cm}
\pagestyle{fancy}
\fancyhfoffset[]{0pt}
\setlength{\headheight}{30pt}
\lhead{\includegraphics[height = 0.7cm]{IMAGES/logo/logo_univ_toulon.png}}
\rhead{\includegraphics[height = 0.9cm]{IMAGES/logo/logo_seatech.png}}
\lfoot{\auteur\;\&\;\auteurs}
% \cfoot{\textcolor{Red}{\textbf{CONFIDENTIAL}}}
\rfoot{Page \thepage \ / \pageref{LastPage}}

\title{\titre}
\author{\auteur \& \auteurs}
\date{\today}

%%%%%%%%%%%%%%%%%%%%%%%%%
% Space around pictures %
%%%%%%%%%%%%%%%%%%%%%%%%%
\newcommand{\customspace}{6pt}
\setlength{\textfloatsep}{\customspace}
\setlength{\intextsep}{\customspace}
\setlength{\floatsep}{\customspace}

%%%%%%%%%%%%%%%%%%
% Hyperref setup %
%%%%%%%%%%%%%%%%%%
\hypersetup{
  colorlinks=true,
  linkcolor=darkgray,
  filecolor=magenta,      
  urlcolor=cyan,
  citecolor=purple,
  pdftitle={mathe_stablo_rapport_ifs},
}

%%%%%%%%%%%%%%%%%%%%%%%%%%%%%%%%%%%
% Excluding (sub)section from ToC %
%%%%%%%%%%%%%%%%%%%%%%%%%%%%%%%%%%%
\setcounter{tocdepth}{2} % Change to 1 to exclude subsections as well
\renewcommand{\familydefault}{\sfdefault}

%%%%%%%%%%%%%%%%%%%%
%     Glossary     %
%%%%%%%%%%%%%%%%%%%%
\ifthenelse{\boolean{gloss}}{
  % \setglossarystyle{tree}

\makeglossaries

\newacronym{ac}{AC}{Acronym}


\newglossaryentry{sfs}
{
    name={\ensuremath{Symbol_{for short}}},
    description={Fast way to talk to something}
}

}


%%%%%%%%%%%%%%%%%%%%%%%%%%%%%%%%%%%%%%%%%%%%%%%%%%%%%%%%%%%%%%%%%%%%%%%%%%%%%%%%%%%%%%%%%%%%%%%%%%%%%%%%%%%%
%                                                                                                          %
%                                             Début du document                                            %
%                                                                                                          %
%%%%%%%%%%%%%%%%%%%%%%%%%%%%%%%%%%%%%%%%%%%%%%%%%%%%%%%%%%%%%%%%%%%%%%%%%%%%%%%%%%%%%%%%%%%%%%%%%%%%%%%%%%%%

\begin{document}

%%%%%%%%%%%%%%%%%%%%%%%%%
%     Page de garde     %
%%%%%%%%%%%%%%%%%%%%%%%%%
\begin{titlepage}
\AddToShipoutPictureBG*{\includegraphics[width=\paperwidth,height=\paperheight]{IMAGES/page_garde.png}}

    \centering
    
    % Espacement vertical
    \vspace*{4cm}
    
    % Barres horizontales
    \makebox[0.7\linewidth]{\hrulefill}\\[0.2cm]
    
    % Titre encadré
    \vspace{0.5cm}
    \begin{minipage}{\textwidth}
        \centering
        {\fontsize{28}{48}\selectfont \textsc{\titre}}\\[0.2cm]
        % {\fontsize{18}{48}\selectfont \textsc{\soustitre}}
    \end{minipage}
    \vspace{0.3cm}
    
    % Barres horizontales
    \makebox[0.8\linewidth]{\hrulefill}\\[0.2cm]
    
    % Espacement vertical
    \vspace{2cm}
    
    % Description
    \large{\Large \textbf{\sujet}}\\
    \large{\textbf{\sujets}}\\

    \vspace{0.5cm}
    \large{\textbf{2024 - 2025}}

    % \begin{mdframed}[backgroundcolor=red!20,linewidth=0pt]
    %   \vspace{0.5cm}
    %   \centering{\textcolor{Red}{\textbf{CONFIDENTIAL}}}
    %   \vspace{0.5cm}
    % \end{mdframed}
    
    \vspace{1cm}
    
    % Encadrants et élèves
    %  \begin{flushright}
    {
    \centering
        Auteurs : \\
        \auteur\;\&\;\auteurs

        \vspace{0.8 cm}

        Professeur encadrant : \\
        \referent

    }
    %  \end{flushright}
    
\end{titlepage}
\ClearShipoutPictureBG
\newpage

% Sommaire
\renewcommand{\thepage}{}

\renewcommand{\thepage}{\arabic{page}}
\renewcommand{\thesection}{\Roman{section}}

% \includepdf{annexes/plagiat.pdf}

%%%%%%%%%%%%%%%%%%
%     Résumé     %
%%%%%%%%%%%%%%%%%%
\ifthenelse{\boolean{abst}}{
  \documentclass[../main.tex]{subfiles}
\LoadClass[a4paper,12pt]{article}
\documentclass{article}
%%%%%%%%%%%%%%%%%%%%%%%%%%%%%%%%%%%%%%%%%%%%%%%%%%%%%%%%%%%%%%%
%                        Packages import                      %
%%%%%%%%%%%%%%%%%%%%%%%%%%%%%%%%%%%%%%%%%%%%%%%%%%%%%%%%%%%%%%%
\input{_HEADERS_/package.tex}

%%%%%%%%%%%%%%%%%%%%%%%%%%%%%%%%%%%%%%%%%%%%%%%%%%%%%%%%%%%%%%%
%                                                             %
%                       Codes & sections                      %
%                                                             %
%%%%%%%%%%%%%%%%%%%%%%%%%%%%%%%%%%%%%%%%%%%%%%%%%%%%%%%%%%%%%%%

% Création des boolean
\newboolean{intro}
\newboolean{sumup}
\newboolean{annexe}
\newboolean{biblio}
\newboolean{abst}
\newboolean{thx}
\newboolean{gloss}
\newboolean{fortran}
\newboolean{cpp}
\newboolean{python}
\newboolean{ffpp}

%% Sections %%
% Renseigner si le Rapport contient une introdcution 
\setboolean{intro}{true}
% Renseigner si le Rapport contient une conclusion 
\setboolean{sumup}{true}
% Renseigner si le Rapport contient des annexes 
\setboolean{annexe}{false}
% Renseigner si le Rapport contient un abstract
\setboolean{abst}{false}
% Renseigner si le Rapport contient des remerciements
\setboolean{thx}{false}
% Renseigner si le document contient une bibliographie
\setboolean{biblio}{false} 
% Renseigner si le document contient un glossaire
\setboolean{gloss}{false}

%% Codes %%
% Renseigner si le Rapport contient des codes fortran
\setboolean{fortran}{false}
% Renseigner si le Rapport contient des codes C++
\setboolean{cpp}{false}
% Renseigner si le Rapport contient des codes Python
\setboolean{python}{false}
% Renseigner si le Rapport contient des codes FreeFEM++
\setboolean{ffpp}{false}

% =============================================================
\ifthenelse{\boolean{fortran}}{ % Mise en page des codes F90
\input{_HEADERS_/display_codes/display_f90.tex}}


% =============================================================
\ifthenelse{\boolean{cpp}}{ % Mise en page des codes C++
\input{_HEADERS_/display_codes/display_cpp.tex}}


% =============================================================
\ifthenelse{\boolean{python}}{  % Mise en page des codes Python
\input{_HEADERS_/display_codes/display_python.tex}}


% =============================================================
\ifthenelse{\boolean{ffpp}}{  % Mise en page des codes FreeFEM++
\input{_HEADERS_/display_codes/display_ffpp.tex}}


%%%%%%%%%%%%%%%%%%%%%%%%%%%%%%%%%%%%%%%%%%%%%%%%%%%%%%%%%%%%%%%
%                      Environments setup                     %
%%%%%%%%%%%%%%%%%%%%%%%%%%%%%%%%%%%%%%%%%%%%%%%%%%%%%%%%%%%%%%%
\input{_HEADERS_/envt}

%%%%%%%%%%%%%%%%%%%%%%%%%%%%%%%%%%%%%%%%%%%%%%%%%%%%%%%%%%%%%%%
%                       Custom commands                       %
%%%%%%%%%%%%%%%%%%%%%%%%%%%%%%%%%%%%%%%%%%%%%%%%%%%%%%%%%%%%%%%
\input{_HEADERS_/cmd.tex}

%%%%%%%%%%%%%%%%%%%%%%%%%%%%%%%%%%%%%%%%%%%%%%%%%%%%%%%%%%%%%%%
%                                                             %
%                      Document settings                      %
%                                                             %
%%%%%%%%%%%%%%%%%%%%%%%%%%%%%%%%%%%%%%%%%%%%%%%%%%%%%%%%%%%%%%%
\numberwithin{figure}{section}
\numberwithin{table}{section}
%%%%%%%%%%%%%%%%%%%%%%%
% Titre, auteurs, ... %
%%%%%%%%%%%%%%%%%%%%%%%
\newcommand{\titre}{Modélisation du trimaran Zephyr\\-\\\vspace{0.5cm} Code de calcul CERF}
\newcommand{\sujet}{Interactions Fluide-Structure}
\newcommand{\sujets}{Seatech 3A - MOCA}
\newcommand{\auteur}{Fabien MATHE}
\newcommand{\auteurs}{Yohan STABLO}
\newcommand{\referent}{M. Frédéric GOLAY}

%%%%%%%%%%%%%%%%%%%%%%%%%%%%%%
% En-têtes en pieds de pages %
%%%%%%%%%%%%%%%%%%%%%%%%%%%%%%
\geometry{hmargin=2cm,vmargin=2.3cm}
\pagestyle{fancy}
\fancyhfoffset[]{0pt}
\setlength{\headheight}{30pt}
\lhead{\includegraphics[height = 0.7cm]{IMAGES/logo/logo_univ_toulon.png}}
\rhead{\includegraphics[height = 0.9cm]{IMAGES/logo/logo_seatech.png}}
\lfoot{\auteur\;\&\;\auteurs}
% \cfoot{\textcolor{Red}{\textbf{CONFIDENTIAL}}}
\rfoot{Page \thepage \ / \pageref{LastPage}}

\title{\titre}
\author{\auteur \& \auteurs}
\date{\today}

%%%%%%%%%%%%%%%%%%%%%%%%%
% Space around pictures %
%%%%%%%%%%%%%%%%%%%%%%%%%
\newcommand{\customspace}{6pt}
\setlength{\textfloatsep}{\customspace}
\setlength{\intextsep}{\customspace}
\setlength{\floatsep}{\customspace}

%%%%%%%%%%%%%%%%%%
% Hyperref setup %
%%%%%%%%%%%%%%%%%%
\hypersetup{
  colorlinks=true,
  linkcolor=darkgray,
  filecolor=magenta,      
  urlcolor=cyan,
  citecolor=purple,
  pdftitle={mathe_stablo_rapport_ifs},
}

%%%%%%%%%%%%%%%%%%%%%%%%%%%%%%%%%%%
% Excluding (sub)section from ToC %
%%%%%%%%%%%%%%%%%%%%%%%%%%%%%%%%%%%
\setcounter{tocdepth}{2} % Change to 1 to exclude subsections as well
\renewcommand{\familydefault}{\sfdefault}

%%%%%%%%%%%%%%%%%%%%
%     Glossary     %
%%%%%%%%%%%%%%%%%%%%
\ifthenelse{\boolean{gloss}}{
  \input{annexes/glossary.tex}
}


%%%%%%%%%%%%%%%%%%%%%%%%%%%%%%%%%%%%%%%%%%%%%%%%%%%%%%%%%%%%%%%%%%%%%%%%%%%%%%%%%%%%%%%%%%%%%%%%%%%%%%%%%%%%
%                                                                                                          %
%                                             Début du document                                            %
%                                                                                                          %
%%%%%%%%%%%%%%%%%%%%%%%%%%%%%%%%%%%%%%%%%%%%%%%%%%%%%%%%%%%%%%%%%%%%%%%%%%%%%%%%%%%%%%%%%%%%%%%%%%%%%%%%%%%%

\begin{document}

%%%%%%%%%%%%%%%%%%%%%%%%%
%     Page de garde     %
%%%%%%%%%%%%%%%%%%%%%%%%%
\begin{titlepage}
\AddToShipoutPictureBG*{\includegraphics[width=\paperwidth,height=\paperheight]{IMAGES/page_garde.png}}

    \centering
    
    % Espacement vertical
    \vspace*{4cm}
    
    % Barres horizontales
    \makebox[0.7\linewidth]{\hrulefill}\\[0.2cm]
    
    % Titre encadré
    \vspace{0.5cm}
    \begin{minipage}{\textwidth}
        \centering
        {\fontsize{28}{48}\selectfont \textsc{\titre}}\\[0.2cm]
        % {\fontsize{18}{48}\selectfont \textsc{\soustitre}}
    \end{minipage}
    \vspace{0.3cm}
    
    % Barres horizontales
    \makebox[0.8\linewidth]{\hrulefill}\\[0.2cm]
    
    % Espacement vertical
    \vspace{2cm}
    
    % Description
    \large{\Large \textbf{\sujet}}\\
    \large{\textbf{\sujets}}\\

    \vspace{0.5cm}
    \large{\textbf{2024 - 2025}}

    % \begin{mdframed}[backgroundcolor=red!20,linewidth=0pt]
    %   \vspace{0.5cm}
    %   \centering{\textcolor{Red}{\textbf{CONFIDENTIAL}}}
    %   \vspace{0.5cm}
    % \end{mdframed}
    
    \vspace{1cm}
    
    % Encadrants et élèves
    %  \begin{flushright}
    {
    \centering
        Auteurs : \\
        \auteur\;\&\;\auteurs

        \vspace{0.8 cm}

        Professeur encadrant : \\
        \referent

    }
    %  \end{flushright}
    
\end{titlepage}
\ClearShipoutPictureBG
\newpage

% Sommaire
\renewcommand{\thepage}{}

\renewcommand{\thepage}{\arabic{page}}
\renewcommand{\thesection}{\Roman{section}}

% \includepdf{annexes/plagiat.pdf}

%%%%%%%%%%%%%%%%%%
%     Résumé     %
%%%%%%%%%%%%%%%%%%
\ifthenelse{\boolean{abst}}{
  \input{SECTIONS/abstract.tex}  
  \newpage
}

%%%%%%%%%%%%%%%%%%%%%%%%%
%     Remerciements     %
%%%%%%%%%%%%%%%%%%%%%%%%%
\ifthenelse{\boolean{thx}}{
  \input{SECTIONS/acknow.tex}
}

%%%%%%%%%%%%%%%%%%%%%%%%%%%%
%     Plan du document     %
%%%%%%%%%%%%%%%%%%%%%%%%%%%%
\vfill
\tableofcontents
\vfill

\newpage

%%%%%%%%%%%%%%
% Glossaries %
%%%%%%%%%%%%%%
\ifthenelse{\boolean{gloss}}{
  \printglossary[type=\acronymtype]
  \newpage
  \printglossary
  \newpage
}

%%%%%%%%%%%%%%%%
% Introduction %
%%%%%%%%%%%%%%%%
\ifthenelse{\boolean{intro}}{
\section*{Introduction}
\addcontentsline{toc}{section}{\protect\numberline{}Introduction}
\subfile{SECTIONS/introduction}
\newpage
}

\section{Initialisation}
\subfile{SECTIONS/part1}

\newpage

\section{Résultats et discussions}
\subfile{SECTIONS/part2}

\newpage

\section{Part 3}
\subfile{SECTIONS/part3}

\newpage

\ifthenelse{\boolean{sumup}}{
\section*{Conclusion}
\addcontentsline{toc}{section}{\protect\numberline{}Conclusion}%
\subfile{SECTIONS/sumup}
\newpage
}

%%%%%%%%%%%%%%%%%%%
%     Annexes     %
%%%%%%%%%%%%%%%%%%%

\ifthenelse{\boolean{annexe}}{
\addcontentsline{toc}{section}{\protect\numberline{}Appendix}%
\vspace*{\stretch{1}}
\begin{center}
  \section*{\LARGE\textsc{Appendix}}
\end{center}
\vspace*{\stretch{1}}
\newpage
\subfile{SECTIONS/appendix.tex}
}

\newpage 

\ifthenelse{\boolean{biblio}}{
\addcontentsline{toc}{section}{\protect\numberline{}References}
\bibliographystyle{unsrt}
\footnotesize{\bibliography{SECTIONS/bib.bib}}
% \nocite{*}
}

\end{document}

\begin{document}
    \section*{}
    \addcontentsline{toc}{section}{\protect\numberline{}Abstract}%
    \vspace*{\stretch{1}}
    % \vfill
    \begin{abstract}
    \fontsize{10}{12}\selectfont
    \noindent
    \textbf{ENGLISH}
    \vspace*{0.2cm}\\

    \vspace*{0.2cm}

    \textbf{Keywords:}\\[0.1cm]
    \textsc{}\\[0.3cm]
    % \newpage
    %%%%%%%%%%%%%%%%
    \noindent
    \textbf{FRENCH}
    \vspace*{0.2cm}\\

    \vspace*{0.2cm}

    \textbf{Mots-clés:}\\[0.1cm]
    \textsc{}\\[0.3cm]

    \end{abstract}
    \vspace*{\stretch{1}}
  % \vfill
\end{document}  
  \newpage
}

%%%%%%%%%%%%%%%%%%%%%%%%%
%     Remerciements     %
%%%%%%%%%%%%%%%%%%%%%%%%%
\ifthenelse{\boolean{thx}}{
  \documentclass[../main.tex]{subfiles}
\LoadClass[a4paper,12pt]{article}
\documentclass{article}
%%%%%%%%%%%%%%%%%%%%%%%%%%%%%%%%%%%%%%%%%%%%%%%%%%%%%%%%%%%%%%%
%                        Packages import                      %
%%%%%%%%%%%%%%%%%%%%%%%%%%%%%%%%%%%%%%%%%%%%%%%%%%%%%%%%%%%%%%%
\input{_HEADERS_/package.tex}

%%%%%%%%%%%%%%%%%%%%%%%%%%%%%%%%%%%%%%%%%%%%%%%%%%%%%%%%%%%%%%%
%                                                             %
%                       Codes & sections                      %
%                                                             %
%%%%%%%%%%%%%%%%%%%%%%%%%%%%%%%%%%%%%%%%%%%%%%%%%%%%%%%%%%%%%%%

% Création des boolean
\newboolean{intro}
\newboolean{sumup}
\newboolean{annexe}
\newboolean{biblio}
\newboolean{abst}
\newboolean{thx}
\newboolean{gloss}
\newboolean{fortran}
\newboolean{cpp}
\newboolean{python}
\newboolean{ffpp}

%% Sections %%
% Renseigner si le Rapport contient une introdcution 
\setboolean{intro}{true}
% Renseigner si le Rapport contient une conclusion 
\setboolean{sumup}{true}
% Renseigner si le Rapport contient des annexes 
\setboolean{annexe}{false}
% Renseigner si le Rapport contient un abstract
\setboolean{abst}{false}
% Renseigner si le Rapport contient des remerciements
\setboolean{thx}{false}
% Renseigner si le document contient une bibliographie
\setboolean{biblio}{false} 
% Renseigner si le document contient un glossaire
\setboolean{gloss}{false}

%% Codes %%
% Renseigner si le Rapport contient des codes fortran
\setboolean{fortran}{false}
% Renseigner si le Rapport contient des codes C++
\setboolean{cpp}{false}
% Renseigner si le Rapport contient des codes Python
\setboolean{python}{false}
% Renseigner si le Rapport contient des codes FreeFEM++
\setboolean{ffpp}{false}

% =============================================================
\ifthenelse{\boolean{fortran}}{ % Mise en page des codes F90
\input{_HEADERS_/display_codes/display_f90.tex}}


% =============================================================
\ifthenelse{\boolean{cpp}}{ % Mise en page des codes C++
\input{_HEADERS_/display_codes/display_cpp.tex}}


% =============================================================
\ifthenelse{\boolean{python}}{  % Mise en page des codes Python
\input{_HEADERS_/display_codes/display_python.tex}}


% =============================================================
\ifthenelse{\boolean{ffpp}}{  % Mise en page des codes FreeFEM++
\input{_HEADERS_/display_codes/display_ffpp.tex}}


%%%%%%%%%%%%%%%%%%%%%%%%%%%%%%%%%%%%%%%%%%%%%%%%%%%%%%%%%%%%%%%
%                      Environments setup                     %
%%%%%%%%%%%%%%%%%%%%%%%%%%%%%%%%%%%%%%%%%%%%%%%%%%%%%%%%%%%%%%%
\input{_HEADERS_/envt}

%%%%%%%%%%%%%%%%%%%%%%%%%%%%%%%%%%%%%%%%%%%%%%%%%%%%%%%%%%%%%%%
%                       Custom commands                       %
%%%%%%%%%%%%%%%%%%%%%%%%%%%%%%%%%%%%%%%%%%%%%%%%%%%%%%%%%%%%%%%
\input{_HEADERS_/cmd.tex}

%%%%%%%%%%%%%%%%%%%%%%%%%%%%%%%%%%%%%%%%%%%%%%%%%%%%%%%%%%%%%%%
%                                                             %
%                      Document settings                      %
%                                                             %
%%%%%%%%%%%%%%%%%%%%%%%%%%%%%%%%%%%%%%%%%%%%%%%%%%%%%%%%%%%%%%%
\numberwithin{figure}{section}
\numberwithin{table}{section}
%%%%%%%%%%%%%%%%%%%%%%%
% Titre, auteurs, ... %
%%%%%%%%%%%%%%%%%%%%%%%
\newcommand{\titre}{Modélisation du trimaran Zephyr\\-\\\vspace{0.5cm} Code de calcul CERF}
\newcommand{\sujet}{Interactions Fluide-Structure}
\newcommand{\sujets}{Seatech 3A - MOCA}
\newcommand{\auteur}{Fabien MATHE}
\newcommand{\auteurs}{Yohan STABLO}
\newcommand{\referent}{M. Frédéric GOLAY}

%%%%%%%%%%%%%%%%%%%%%%%%%%%%%%
% En-têtes en pieds de pages %
%%%%%%%%%%%%%%%%%%%%%%%%%%%%%%
\geometry{hmargin=2cm,vmargin=2.3cm}
\pagestyle{fancy}
\fancyhfoffset[]{0pt}
\setlength{\headheight}{30pt}
\lhead{\includegraphics[height = 0.7cm]{IMAGES/logo/logo_univ_toulon.png}}
\rhead{\includegraphics[height = 0.9cm]{IMAGES/logo/logo_seatech.png}}
\lfoot{\auteur\;\&\;\auteurs}
% \cfoot{\textcolor{Red}{\textbf{CONFIDENTIAL}}}
\rfoot{Page \thepage \ / \pageref{LastPage}}

\title{\titre}
\author{\auteur \& \auteurs}
\date{\today}

%%%%%%%%%%%%%%%%%%%%%%%%%
% Space around pictures %
%%%%%%%%%%%%%%%%%%%%%%%%%
\newcommand{\customspace}{6pt}
\setlength{\textfloatsep}{\customspace}
\setlength{\intextsep}{\customspace}
\setlength{\floatsep}{\customspace}

%%%%%%%%%%%%%%%%%%
% Hyperref setup %
%%%%%%%%%%%%%%%%%%
\hypersetup{
  colorlinks=true,
  linkcolor=darkgray,
  filecolor=magenta,      
  urlcolor=cyan,
  citecolor=purple,
  pdftitle={mathe_stablo_rapport_ifs},
}

%%%%%%%%%%%%%%%%%%%%%%%%%%%%%%%%%%%
% Excluding (sub)section from ToC %
%%%%%%%%%%%%%%%%%%%%%%%%%%%%%%%%%%%
\setcounter{tocdepth}{2} % Change to 1 to exclude subsections as well
\renewcommand{\familydefault}{\sfdefault}

%%%%%%%%%%%%%%%%%%%%
%     Glossary     %
%%%%%%%%%%%%%%%%%%%%
\ifthenelse{\boolean{gloss}}{
  \input{annexes/glossary.tex}
}


%%%%%%%%%%%%%%%%%%%%%%%%%%%%%%%%%%%%%%%%%%%%%%%%%%%%%%%%%%%%%%%%%%%%%%%%%%%%%%%%%%%%%%%%%%%%%%%%%%%%%%%%%%%%
%                                                                                                          %
%                                             Début du document                                            %
%                                                                                                          %
%%%%%%%%%%%%%%%%%%%%%%%%%%%%%%%%%%%%%%%%%%%%%%%%%%%%%%%%%%%%%%%%%%%%%%%%%%%%%%%%%%%%%%%%%%%%%%%%%%%%%%%%%%%%

\begin{document}

%%%%%%%%%%%%%%%%%%%%%%%%%
%     Page de garde     %
%%%%%%%%%%%%%%%%%%%%%%%%%
\begin{titlepage}
\AddToShipoutPictureBG*{\includegraphics[width=\paperwidth,height=\paperheight]{IMAGES/page_garde.png}}

    \centering
    
    % Espacement vertical
    \vspace*{4cm}
    
    % Barres horizontales
    \makebox[0.7\linewidth]{\hrulefill}\\[0.2cm]
    
    % Titre encadré
    \vspace{0.5cm}
    \begin{minipage}{\textwidth}
        \centering
        {\fontsize{28}{48}\selectfont \textsc{\titre}}\\[0.2cm]
        % {\fontsize{18}{48}\selectfont \textsc{\soustitre}}
    \end{minipage}
    \vspace{0.3cm}
    
    % Barres horizontales
    \makebox[0.8\linewidth]{\hrulefill}\\[0.2cm]
    
    % Espacement vertical
    \vspace{2cm}
    
    % Description
    \large{\Large \textbf{\sujet}}\\
    \large{\textbf{\sujets}}\\

    \vspace{0.5cm}
    \large{\textbf{2024 - 2025}}

    % \begin{mdframed}[backgroundcolor=red!20,linewidth=0pt]
    %   \vspace{0.5cm}
    %   \centering{\textcolor{Red}{\textbf{CONFIDENTIAL}}}
    %   \vspace{0.5cm}
    % \end{mdframed}
    
    \vspace{1cm}
    
    % Encadrants et élèves
    %  \begin{flushright}
    {
    \centering
        Auteurs : \\
        \auteur\;\&\;\auteurs

        \vspace{0.8 cm}

        Professeur encadrant : \\
        \referent

    }
    %  \end{flushright}
    
\end{titlepage}
\ClearShipoutPictureBG
\newpage

% Sommaire
\renewcommand{\thepage}{}

\renewcommand{\thepage}{\arabic{page}}
\renewcommand{\thesection}{\Roman{section}}

% \includepdf{annexes/plagiat.pdf}

%%%%%%%%%%%%%%%%%%
%     Résumé     %
%%%%%%%%%%%%%%%%%%
\ifthenelse{\boolean{abst}}{
  \input{SECTIONS/abstract.tex}  
  \newpage
}

%%%%%%%%%%%%%%%%%%%%%%%%%
%     Remerciements     %
%%%%%%%%%%%%%%%%%%%%%%%%%
\ifthenelse{\boolean{thx}}{
  \input{SECTIONS/acknow.tex}
}

%%%%%%%%%%%%%%%%%%%%%%%%%%%%
%     Plan du document     %
%%%%%%%%%%%%%%%%%%%%%%%%%%%%
\vfill
\tableofcontents
\vfill

\newpage

%%%%%%%%%%%%%%
% Glossaries %
%%%%%%%%%%%%%%
\ifthenelse{\boolean{gloss}}{
  \printglossary[type=\acronymtype]
  \newpage
  \printglossary
  \newpage
}

%%%%%%%%%%%%%%%%
% Introduction %
%%%%%%%%%%%%%%%%
\ifthenelse{\boolean{intro}}{
\section*{Introduction}
\addcontentsline{toc}{section}{\protect\numberline{}Introduction}
\subfile{SECTIONS/introduction}
\newpage
}

\section{Initialisation}
\subfile{SECTIONS/part1}

\newpage

\section{Résultats et discussions}
\subfile{SECTIONS/part2}

\newpage

\section{Part 3}
\subfile{SECTIONS/part3}

\newpage

\ifthenelse{\boolean{sumup}}{
\section*{Conclusion}
\addcontentsline{toc}{section}{\protect\numberline{}Conclusion}%
\subfile{SECTIONS/sumup}
\newpage
}

%%%%%%%%%%%%%%%%%%%
%     Annexes     %
%%%%%%%%%%%%%%%%%%%

\ifthenelse{\boolean{annexe}}{
\addcontentsline{toc}{section}{\protect\numberline{}Appendix}%
\vspace*{\stretch{1}}
\begin{center}
  \section*{\LARGE\textsc{Appendix}}
\end{center}
\vspace*{\stretch{1}}
\newpage
\subfile{SECTIONS/appendix.tex}
}

\newpage 

\ifthenelse{\boolean{biblio}}{
\addcontentsline{toc}{section}{\protect\numberline{}References}
\bibliographystyle{unsrt}
\footnotesize{\bibliography{SECTIONS/bib.bib}}
% \nocite{*}
}

\end{document}

\begin{document}
    \addcontentsline{toc}{section}{\protect\numberline{}Acknowledgment}%

    \vspace*{\stretch{1}}
    \begin{center}
    {\large \textbf{Acknowledgment}}\\[0.8cm]

    \end{center}
    \vspace*{\stretch{1}}
    \newpage
\end{document}
}

%%%%%%%%%%%%%%%%%%%%%%%%%%%%
%     Plan du document     %
%%%%%%%%%%%%%%%%%%%%%%%%%%%%
\vfill
\tableofcontents
\vfill

\newpage

%%%%%%%%%%%%%%
% Glossaries %
%%%%%%%%%%%%%%
\ifthenelse{\boolean{gloss}}{
  \printglossary[type=\acronymtype]
  \newpage
  \printglossary
  \newpage
}

%%%%%%%%%%%%%%%%
% Introduction %
%%%%%%%%%%%%%%%%
\ifthenelse{\boolean{intro}}{
\section*{Introduction}
\addcontentsline{toc}{section}{\protect\numberline{}Introduction}
\subfile{SECTIONS/introduction}
\newpage
}

\section{Initialisation}
\subfile{SECTIONS/part1}

\newpage

\section{Résultats et discussions}
\subfile{SECTIONS/part2}

\newpage

\section{Part 3}
\subfile{SECTIONS/part3}

\newpage

\ifthenelse{\boolean{sumup}}{
\section*{Conclusion}
\addcontentsline{toc}{section}{\protect\numberline{}Conclusion}%
\subfile{SECTIONS/sumup}
\newpage
}

%%%%%%%%%%%%%%%%%%%
%     Annexes     %
%%%%%%%%%%%%%%%%%%%

\ifthenelse{\boolean{annexe}}{
\addcontentsline{toc}{section}{\protect\numberline{}Appendix}%
\vspace*{\stretch{1}}
\begin{center}
  \section*{\LARGE\textsc{Appendix}}
\end{center}
\vspace*{\stretch{1}}
\newpage
\subfile{SECTIONS/appendix.tex}
}

\newpage 

\ifthenelse{\boolean{biblio}}{
\addcontentsline{toc}{section}{\protect\numberline{}References}
\bibliographystyle{unsrt}
\footnotesize{\bibliography{SECTIONS/bib.bib}}
% \nocite{*}
}

\end{document}

\begin{document}
Ce projet nous a permis de découvrir et de tester le code CERF sur des cas concrets autour du trimaran Zephyr.\\
Les simulations ont montré des résultats intéressants, mais aussi des limites, comme le besoin d'un maillage toujours plus précis ou les contraintes de temps de calcul.\\
De plus, plusieurs problèmes ont été soulevés concernant notament l'interaction autour des objets.

Ce projet nous a permis de découvrir et de tester le code CERF à travers des simulations appliquées au trimaran Zephyr. Les cas étudiés, allant de la mise à l'eau à la submersion par une vague, ont permis d'obtenir des résultats pertinents tout en posant certaines limites. Parmi celles-ci, le besoin d'un maillage toujours plus précis pour capturer correctement les interfaces fluide-structure, ainsi que les contraintes importantes liées au temps de calcul, ont été les plus fréquentes.

En outre, des problématiques spécifiques, notamment concernant l'interaction entre le fluide et les objets solides, ont été soulevées.

\end{document}
